Det er i løbet af projektet blevet vurderet hvordan RSA er vigtig i forbindelse med sikkerheden ved logins.
Først ved en redegørelse af talteorien bag RSA, dernæst en kort gennemgang af hashingmetoder.
Så blev der analyseret på forskellen mellem kryptografi og hashing, for at runde af med en gennemgang af RSA protokollen og et konkreteksempel heraf.
\par
Ud fra dette, kan det konkluderes at RSA er en del af hverdagen, for alle som bruger internettet dagligt.
Hvis der en dag bliver fremskrevet et bevis for, hvordan man nemt kan primtalsfaktorisere et stort tal, vil hele vores verden formentlig gå i stå, da stort set alt hvad vi laver i dag, har forbindelse til internettet på den ene eller den anden måde.

Samtidig er det vist hvorfor det er vigtigt at være opmærksom på sikkerheden, ved de hjemmesider man besøger.
Som bruger har man ikke så stor indflydelse på sikkerheden, hvorfor det er godt at holde øje med tegn på sårbarheder, såsom brug af \texttt{http} (ikke-krypteret forbindelse) i stedet for \texttt{https} (krypteret forbindelse).

Som administrator af en hjemmeside er det vigtigt at sørge for at have sikkerheden i orden, dette gælder både kryptering såvel som hashing.

I den forbindelse, er RSA altså voldsomt vigtig, vores verden ville ikke være den samme i dag, hvis det ikke var for \textit{Rivers}, \textit{Shamir} og \textit{Adleman}.

% \emph{RSA er altså uundgåelig foranstaltning, hvis man gerne vil leve i den digitale verden, som vi nu en gang gør.}
