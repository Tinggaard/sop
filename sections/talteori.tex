Talteorien beskæftiger sig med alle de hele tal \(\mathbb{Z}\).
Vi skal i løbet af dette afsnit kigge på hvordan talteori ligger til grunde for, at RSA virker i praksis.
\\
Det er ikke alle definitioner og sætninger der bliver bevist, da det ville tage for lang tid.
Alle beviser kan naturligvis findes i \textit{Kryptografi -- fra viden til videnskab}. \cite{krypto}



    \subsection{Division med rest og divisor}
    I stedet for at regne brøker som rationelle mængder \(\mathbb{Q}\) eller decimaltal, skal vi beskæftige os med hvordan man regner med rester.
    Det vil altså sige at vi udelukkende vil beskæftige os med heltal \(\mathbb{Z}\).

    Et tal \(a \neq 0\) kan enten gå op i \(b\) eller ikke gå op i \(b\).
    Vi definerer at \(a\) går op i \(b\), hvis der ingen rest er, efter division. Eller sagt på en anden måde:


    \begin{definition}
        \label{heldiv}
        Givet tallet \(a \neq 0\) og tallet \(b \geq a\), vil \(a\) gå op i \(b\),
        hvis der findes et heltal \(q\), som opfylder \(b = q \cdot a\).\\
        Det skrives \(a \mid b\), som betyder at \(a\) går op i \(b\).\\
        \(a\) betenes som \textit{divisoren} eller en \textit{faktor},
        % \(b\) betegnes \textit{dividenden},
        mens \(q\) kaldes for \textit{kvotienten}.\cite[70]{krypto}
    \end{definition}

    Er der i stedet en rest efter division, vil man kunne skrive tallet som \(a \nmid b\), som betyder at \(a\) ikke går op i \(b\).
    % Det at arbejde med rester er også noget vi skal beskæftige os meget med.

    \begin{sent}
        \label{rest}
        Givet tallet \(a > 0\) og tallet \(b\), findes der specifikke tal \(q\) og \(r\), som opfylder \(b = q \cdot a + r\), hvor \(0 \leq r < q\). \(r\) betegnes som \textit{resten}.
    \end{sent}

    Det ses ved nogle eksempler

    \begin{eks}
        Lad \(a = 48\) og \(b = 8\), det ses da at \(q = 6\) og \(r = 0\), da \(48 = 6 \cdot 8 + 0\) og da \(r = 0\), medfører det at \(a \mid b\).\\

        Lad \(a = 17\) og \(b = 4\), det ses da at \(q = 4\) og \(r = 1\), da \(17 = 4 \cdot 4 + 1\) og da \(r \neq 0\), medfører det at \(17 \nmid 4\).\\

        Man kan også udregne resten ud fra et decimaltal.
        Trykker man \(17 \div 4\) ind på lommeregneren får man \(4.25\), man ganger så decimal delen med \(b\), vil man få resten efter divisionen.
        I dette tilfælde får man \(r = 0.25 \cdot 4 = 1\).
    \end{eks}




    \begin{sent}
        Givet tallene \(a, b, c \in \mathbb{Z}\), hvor \(a, c \neq 0\), vil \(a \mid b\) medføre at \(a c \mid b c\).
    \end{sent}

    \begin{proof}
        Beviset tager selvfølgelig udgangspunkt i definition \ref{heldiv}.
        Da \(a \mid b \implies b = q \cdot a\), vil man ved at gange \(c\) ind, på begge sider, blot få \(b c = q \cdot a c\).
        Og da \(c\) indgår på begge sider af lighedstegnet, kan dette fjernes fra udtrkket, hvorved man er tilbage ved \(b = q \cdot a\).
    \end{proof}





    \begin{sent}
        Givet tallene \(a, b, c \in \mathbb{Z}\), hvor \(a, c \neq 0\), vil \(a \mid b \land b \mid c\) medføre at \(a \mid c\).
    \end{sent}

    \begin{proof}
        \(a \mid b\) skrives som \(b = q_1 \cdot a\), mens \(b \mid c\) skrives som \(c = q_2 \cdot b\). Ved at substituere \(b\) med udtyrkket fra \(a \mid b\), haves \(a = q_1 \cdot q_2 \cdot c\). \(a \mid c\) er altså sand hvis kvotienten er \(q_1 \cdot q_2\).
    \end{proof}


    \begin{eks}
        Her skal gerne være lidt eksempler for at spice det lidt op...
    \end{eks}



    \subsubsection{Modulo}
    Når man laver division med heltal, vil man ofte kigge på resten efter division.
    Derfor er der naturligvis lavet en matematisk operator til at udregne rest efter division -- den kaldes for \emph{modulo}.
    Skriver man f.eks. \(a \pmod{n}\), udregner man resten efter division, som angivet i sætning \ref{rest}.

    \begin{definition}
        For \(m \in \mathbb{Z}\) og \(n \in \mathbb{N}\) gælder:
        \[n \mid m \iff m \pmod{n} = 0\]
        \[r = n - q m = m \pmod{n}\]\cite[72]{krypto}
    \end{definition}

    \begin{eks}
        Et par eksempler på brugen af modulo.
        \begin{center}
            \setlength{\tabcolsep}{20pt} % General space between cols (6pt standard)
            \begin{tabular}{l c r}
                \(43 \pmod{6} = 1\) & \(74 \pmod{10} = 4\) & \(58 \pmod{6} = 4\)
            \end{tabular}
        \end{center}
    \end{eks}


    % sent 4.7 fra bogen (s 73)


    \subsection{Fælles divisor}
    En fællesvidisor bruges til at sige noget om sammenhængen mellem 2 tal.

    \begin{definition}
        For tallene \(a, b, d \in \mathbb{Z}\), betegnes \(d\) som en \emph{fælles divisor}, hvis \(d \mid a \land d \mid b\).
    \end{definition}

    \(a\) og \(b\) vil have et endeligt antal fælles divisorer, da et tal højere end enten \(a\) eller \(b\) naturligvis ikke er en fælles divisor (medmindre \(a\) og \(b\) begge er \(0\).
    Det kan altså konkluderes at der findes en største fælles divisor.

    \begin{definition}[Største fælles divisor]
        Den største fælles divisor for tallene \(a\) og \(b\), betegnes som \((a, b)\).
    \end{definition}

    \begin{eks}
        Lad \(a = 8\) og \(b = 12\).\\
        \begin{tabular*}{\textwidth}{@{} l l}
            Det ses at divisorne i \(8\) er: & \(\pm 1, \pm 2, \pm 4, \pm 8\).\\
            Og det ses at divisorne i \(12\) er:  &\(\pm 1, \pm 2, \pm 3, \pm 4, \pm 6, \pm 12\).\\
            Til fælles har de altså divisorne: & \(\pm 1, \pm 2, \pm 4\).
        \end{tabular*}

        Dermed bliver \((8, 12) = 4\).
    \end{eks}

    %\paragraph{Observationer}
    % Det ses at fortegnet for både \(a\) og \(b\) er ubetydeligt, da kvotienterne kan være negative.

    \begin{definition}[Indbyrdes primiske tal]
        Hvis \((a, b) = 1\), siges \(a\) og \(b\) at være indbyrdes primiske.
    \end{definition}

    Som vist, kan være bøvlet at udregne fælles divisorer for 2 tal, derfor kommer nu en sætning, til at mindske vanskeligheden lidt.

    \begin{sent}
        \label{commod}
        \((a, b) = (a, b \pmod{a})\)
    \end{sent}

    Sætning \ref{commod} bevises ikke, i stedet vil Euclids algoritme (som bygger på sætningen) blive vist.

    \subsubsection{Euclids algoritme}
    Euclids algoritme er essentielt gentagen anvendelse af sætning \ref{commod}.
    \begin{definition}[Euclids algoritme]
        Lad \(a, b \in \mathbb{N}\), hvor \(a \geq b\).
        Hvis \(a \mid b\), så er \((a, b) = b\).
        Hvis \(a \nmid b\), bruges følgende algoritme:
        \begin{align*}
            a   &= b q_0    + r_0   & 0 &<    r_0 < b\\
            b   &= r_0 q_1  + r_1   & 0 &\leq r_1 < r_0\\
            r_0 &= r_1 q_2  + r_2   & 0 &\leq r_2 < r_1\\
            r_1 &= r_2 q_3  + r_3   & 0 &\leq r_3 < r_2\\
            &\vdots
        \end{align*}
        Denne proces slutter, idet der findes en rest på 0.
        Det vil ske efter et endeligt antal gennemløb \(t\).

        \begin{align*}
            r_{t-2} &= r_{t-1}  q_t      + r_t   & 0 &\leq r_t < r_{t-1}\\
            r_{t-1} &= r_t      q_{t+1}  + 0
        \end{align*}
        Den sidste rest, som ikke er 0, er dermed den største fælles divisor.\cite[11]{absalg}
    \end{definition}

    Det ses hvordan Euclids algoritme bruges, til at overskue udregningen af største fælles divisor.

    \begin{eks}[Anvendelse af Euclids algoritme]
        Lad \(a = 1048\) og \(b = 672\).
        Ved brug af Euclids algoritme, findes da største fællesnævner.
        \begin{center}
            \setlength{\tabcolsep}{5pt} % General space between cols (6pt standard)
            \begin{tabular}{r l c r l c r}
                \((1048, 672)\) & \(= (376, 672)\) & idet & \(1048\) & \(= 1 \cdot 672\) & \(+\) & \(376\)\\
                                & \(= (376, 296)\) & --   & \(672\) & \(= 1 \cdot 376\)  & \(+\) & \(296\)\\
                                & \(= (80, 296)\)  & --   & \(376\) & \(= 1 \cdot 296\)  & \(+\) & \(80\)\\
                                & \(= (80, 56)\)   & --   & \(296\) & \(= 3 \cdot 80\)   & \(+\) & \(56\)\\
                                & \(= (24, 56)\)   & --   & \(80\)  & \(= 1 \cdot 56\)   & \(+\) & \(24\)\\
                                & \(= (24, 8)\)    & --   & \(56\)  & \(= 2 \cdot 24\)   & \(+\) & \(8\)\\
                                & \(= (0, 8)\)     & --   & \(24\)  & \(= 3 \cdot 8\)    & \(+\) & \(0\)\\
                                & \(= 8\) & & & & &
            \end{tabular}
        \end{center}
    \end{eks}
    Bilag \ref{file:euclid}, viser en implementering af Euclids algoritme i Python og kan køres således
    (kræver at Python er installeret\footnote{\url{https://www.python.org/downloads/}}):
    \begin{minted}[fontsize=\footnotesize, bgcolor=bg]{bash}
$ python euclid.py a b
    \end{minted}


    \subsection{Eulers \texorpdfstring{\(\phi\)}{Lg}-funktion}
    Euler fremskrev også en funktion, der ud fra et gevent tal \(n\), kan bestemme mængden af tal, som er indbyrdes primiske med \(n\).
    Først betragtes mængden \(\mathbb{Z}_n\). Som er angivet ud fra \(n \in \mathbb{N}\).
    \[\mathbb{Z}_n = \{0, 1, 2, 3, \hdots, n-1 \}\]

    Dette sæt repræsenterer nu alle de mulige rester efter division med \(n\), altså \(\pmod{n}\).
    Der laves nu et nyt sæt bestående udelukkende af tallene, som er indbyrdes primiske med \(n\).
    \[\mathbb{Z}_n^* = \{a \in \mathbb{Z}_n | a \neq 0 \land (a, n) = 1\}\]

    Eulers \(\phi\)-funktion, siges nu at være antallet af elementer i sættet \(\mathbb{Z}_n^*\).

    \begin{eks}
        \mbox{}\vspace*{-1.5em}
        \begin{align*}
            n &= 14 & \mathbb{Z}_{14}^*  &= \{1, 3, 5, 9, 11, 13 \} & \phi(14) &= 6\\
            n &= 9  & \mathbb{Z}_9^*     &= \{1, 2, 4, 5, 7, 8\}    & \phi(9)  &= 6
        \end{align*}
    \end{eks}






    \subsection{Primtal}
    Primtal viser sig at være interessante, i beskæftigelsen med talteori og deivisoere.

    \begin{definition}[Primtal]
        Primtal er heltal, større end \(1\), som kun går op i sig selv og \(1\).
    \end{definition}

    % Vi skal først kigge på en sætning om primtal og division.
    %
    % \begin{sent}
    %     For primtallet \(p\) gælder: \(p \mid a b \implies p \mid a \lor p \mid b\)
    % \end{sent}



    \begin{sent}[Algebraens fundamentalsætning]
        Alle positive heltal, kan skrives som et produkt af primtal.
        Denne faktorisering er unik, hvis man ser bort fra rækkefølgen af primtallene.
    \end{sent}

    \begin{proof}
        Hvis
    \end{proof}


    Derpå bygges en sætning om primtal.

    \begin{sent}
        Der findes uendeligt mange primtal.
    \end{sent}

    \begin{proof}

    \end{proof}
