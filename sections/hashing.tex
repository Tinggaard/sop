Hashing er en metode, brugt til at omdanne data til en tekststreng, med en fikseret længde.
Den nye tekststreng vil være \emph{pseudorandom}\footnote{Fra engelsk: Fremstå tilfældigt, men i virkeligheden være forudsætlig.} % -- den vil fremstå tilfældigt, men faktisk være regelmæssig, baseret på inputtet.
Hashing er ikke ``hemmelig'', forstået ved, at alle kan hashe data og få det samme output (ved samme input naturligvis).
Outputtet efter en hashfunktion, omtales som et \emph{hash} (eller \emph{digest} på engelsk).
\\
Derudover er hashing en envejsfunktion, hvilket betyder, at man altså ikke gå baglæns, hvis man kun har et digest af noget data.\cite{algoritmer}


\subsection{Hashfunktioner}
Der findes en del forskellige hashfunktioner, mange af dem er også implementeret i Python biblioteket\footnote{I computerterminologi er et bibliotek, noget kode, som andre har skrevet som kan implementeres i brugerens egen kode.} ``hashlib''\footnote{\url{https://docs.python.org/3/library/hashlib.html}}.
De forskellige hashfunktioner varierer i længden af output og dermed også hastighed.
Den efterhånden gamle hashfunktion \texttt{MD5}, giver et output på 128-bit, mens nyere funktioner som \texttt{SHA256} og \texttt{SHA384} giver henholdsvis 256 og 384-bit, som navnene antyder.
Det skal senere ses, hvorfor en hashfunktion med et langt output er bedre.

Ønsker man at hashe \(abc\) i \texttt{SHA256} i Python, gøres følgende.
\begin{minted}[fontsize=\footnotesize, bgcolor=bg]{python}
>>> import hashlib
>>> hashlib.sha256('abc'.encode()).hexdigest()
'ba7816bf8f01cfea414140de5dae2223b00361a396177a9cb410ff61f20015ad'
\end{minted}


\subsection{Fordelen ved hashing}
Det at hashing kun går en vej, virker måske lidt nytteløst, da man ikke længere er i stand til at finde ud af hvad man har hashet.
Det man gør, er at man bruger man hashing til at gemme kodeord i en database, sådan at hvis nogle kodeord bliver lækket, er de stadig beskyttede, da de er hashede.\cite{version2}
% I afsnit \ref{kodeord}, vises det hvordan dette er anvendt i praksis.
% Det vil altså sige at hvis man hasher \(abc\), vil man få det samme output hver gang.
% Da et hash er pseudorandom, vil det ikke sige noget om hvad det tidligere har været.

Måden man anvender dette på, er at hashe et kodeord inden det bliver puttet ind i databasen.
Når så en bruger forsøger at logge ind, hasher man igen det indtastede kodeord og hvis hashet stemmer over ens, med det liggende i databasen, kan man formode at brugeren har indtastet det rigtige kodeord.

\begin{eks}[Hashing ved brug af hashfunktionen \texttt{MD5}]
    \label{eks:hash}
    Det ses at, på trods af lignende input, er hashet helt forskelligt.
    \begin{center}
        \texttt{MD5(abc) = 900150983cd24fb0d6963f7d28e17f72}\\
        \texttt{MD5(Abc) = 35593b7ce5020eae3ca68fd5b6f3e031}\\
        \texttt{MD5(abd) = 4911e516e5aa21d327512e0c8b197616}\\
    \end{center}
\end{eks}
% \par
For at forstå hvorfor en langsom hashfunktion er god, er det vigtigt at forstå essensen i en hashfunktion.
Det at den kun kan bruges en vej, gør nemlig at hvis man gerne vil finde ud af den originale tekst til et hash, er man nødt til at prøve sig frem, i computerterminologi kaldes dette for \emph{brute-force}\footnote{Når man prøver alle kobinationer af en fremgangsmetode, for at finde frem til det rigtige svar.}.


\begin{eks}[Kompleksiteten ved brute-force]
    \label{brutefc}
    Antag at man har et hash, hashet i \texttt{MD5}, hvor man gerne vil finde den oprindelige tekst.
    Man ved at teksten består af 8 tegn -- som kan være både store og små bogstaver samt tal.
    I det danske alfabet er der \(29\) bogstaver og disse kan både fremstå stor og små \(2 \cdot 29 = 58\), derudover er der 10 forskellige tal at tage hensyn til \(58 + 10 = 68\).
    Da alle tegn har mulighed for at stå på alle 8 pladser, er der nu \(68^8\approx 4.572 \cdot 10^{14}\) unikke kombinationer.
    En moderne computer kan udregne omkring \(4 \cdot 10^9\) hashes i sekundet, i \texttt{MD5}.
    Mens en lille server kan yde omkring 10 gange så meget \(4 \cdot 10^{10}\).\cite{ytpwd}
    Det udregnes hvor mange sekunder det vil tage at hashe alle kombinationer af 8 bogstaver og tal, for henholdsvis en computer og en lille server.\\
    Det er dog vigtigt at understrege, at man i dag aldrig bør bruge \texttt{MD5}, da metoden er gammel og har vist sige at være for svag til nudagens hardware.\footnote{\url{https://www.kb.cert.org/vuls/id/836068/}}

    \setlength{\tabcolsep}{50pt} % General space between cols (6pt standard)
    \begin{center}
        %\renewcommand{\arraystretch}{1.5} % General space between rows (1 standard)
        \begin{tabular}{c c}

            \(\frac{4.572 \cdot 10^{14}}{4 \cdot 10^9} = 114300\) &
            \(\frac{4.572 \cdot 10^{14}}{4 \cdot 10^{10}} = 11430\)\\

            \multicolumn{2}{c}{Det omskrives til timer.}\\

            \(\frac{114300}{60 \cdot 60} = 31.75\) &
            \(\frac{11430}{60 \cdot 60} = 3.175\)\\

        \end{tabular}
    \end{center}

    Det kan altså konkluderes at det er klogt at vælge et kodeord på mere end 8 tegn.
    Samt at en god hashfunktion tager lang tid, hvorved det forstås at funktioner som \texttt{SHA256} og \texttt{SHA512} er at foretrække over \texttt{MD5}.\\
    Derudover har en hashfunktion med et større output, minde kollisionssandsynlighed\footnote{Når 2 forskellige inputs, giver samme hash-digest.} hvilket også er ønsket, for ikke at kunne danne ens hashes, ud fra forskellige inputs.
\end{eks}



Det skal senere ses, hvordan man kan brute-force endnu længere hashes, uden at det tager helt så lang tid.
Samt hvad man gør for at bekæmpe dette.

\subsection{Hashtabeller}
En anden ting, som hashfunktioner kan bruges til, er at generere hashtabeller.
Dette vil blot blive kort gennemgået, da det ikke er den del af hashfunktioner, som har betydning for onlinesikkerhed.\\
En hashtabel kan bruges idet man har en masse (hashbart) data, som skal sættes op i en tabel.
Når man så skal finde noget af dette data tilbage i tabellen efter at have indsat det, vil man normalt skulle gennemgå alle cellerne i tabellen, for at finde ud af hvor, netop hvor det data man søger er. Det er her hashtabeller kommer ind i billedet.\\
Meget forsimplet, fungerer en hashtabel ved, at man -- inden man insætter dataet i tabellen -- hasher dataet og i stedet for at indsætte det i den næste ledige celle, indsætter man det i en celle, svarende til det hash som dataet genererede.
Når man så vil tilgå noget data i tabel, hasher man blot dataet igen, for så at lave et opslag i tabellen i den celle tilsvarende til hashet. På den måde sparer man en masse tid, ved ikke at skulle lave et gennemløb af alle celler i tabellen.\cite{hashtables}
