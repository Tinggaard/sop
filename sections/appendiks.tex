\section{Beviser} \label{proofs}
\subsection{Bevis for sætning \ref{invelm}}\label{invelemproof}
\textit{Beviset bygger på yderligere ikke-beviste sætninger, som lades op til læseren at validere}
\begin{proof}
    Antag at \((a, n) = 1\), der findes nu en \textit{linearkombination} af \(a\) og \(n\), således at \(a s + n t = 1\),
    hvilket omskrives til \(a s = 1 - n t\).
    Ud fra dette, ses det at \(a s \Mod{n} = 1\).
    Ifølge sætning \ref{modmod}, kan dette omskrives til \(\left(a \Mod{n} \cdot s \Mod{n}\right) \Mod{n} = 1\).
    Det vides at \(a < n\), da \(a \in Z_n\), hvorfor \(a \Mod{n} = a\),
    derfor er \(\left(a s \Mod{n}\right) \Mod{n} = 1\),
    hvilket medfører at \(x = s \Mod{n}\) er løsning til ligningen \(a x \Mod{n} = 1\).
    
    Nu mangler blot entydigheden af beviset.
    Antag at begge løsningerne \(x_1, x_2 \in Z_n\), er løsninger til ligningen \(a x \Mod{n} = 1\).
    Det vil altså sige \(a x_1 \Mod{n} = a x_2 \Mod{n}\),
    ud fra dette kan det konkluderes at \(x_1 \Mod{n} = x_2 \Mod{n}\)
    og da begge løsninger er mindre end \(n\), haves at \(x_1 = x_2\).
    Hvorved entydigheden er vist.

    Sætningen bevises også den anden vej:
    Antag at \(x\) er løsning til ligningen \(a x \Mod{n} = 1\),
    dette medfører at der findes et tal \(q\), så \(a x = q \cdot n + 1\).
    Ud fra dette findes en linearkombination af \(a\) og \(n\) som giver 1:
    \(x \cdot a + (-q) \cdot n = 1\).
    Ud fra en anden ikke-bevist sætning, medfører dette at \((a, n) = 1\)

\end{proof}


\newpage
\section{Python kode}
\subsection{euclid.py}
\label{file:euclid}
Filen ligger også vedhæftet i denne pdf, åbn den ved at dobbeltklikke på ikonet.
\attachfile[description={En implementering af Euclids algoritme i Python.}]{src/euclid.py}
\inputminted[python3, linenos, breaklines, frame=lines, fontsize=\footnotesize]{python}{src/euclid.py}


\subsection{prime\_factor.py}
\label{file:prime_factor}
Filen ligger også vedhæftet i denne pdf, åbn den ved at dobbeltklikke på ikonet.
\attachfile[description={Et script til at lave primtalsfaktorisering.}]{src/prime_factor.py}
\inputminted[python3, linenos, breaklines, frame=lines, fontsize=\footnotesize]{python}{src/prime_factor.py}


\subsection{phi\_func.py}
\label{file:phi_func}
Filen ligger også vedhæftet i denne pdf, åbn den ved at dobbeltklikke på ikonet.
\attachfile[description={Eulers phi-funktion, skrevet i Python.}]{src/phi_func.py}
\inputminted[python3, linenos, breaklines, frame=lines, fontsize=\footnotesize]{python}{src/phi_func.py}


\subsection{eulers\_sent.py}
\label{file:eulers_sent}
Filen ligger også vedhæftet i denne pdf, åbn den ved at dobbeltklikke på ikonet.
\attachfile[description={Et script til at udregne inverse elementer.}]{src/eulers_sent.py}
\inputminted[python3, linenos, breaklines, frame=lines, fontsize=\footnotesize]{python}{src/eulers_sent.py}
