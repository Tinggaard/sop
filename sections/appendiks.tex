\section{Beviser} \label{proofs}
\subsection{Bevis for sætning \ref{invelm}}\label{invelemproof}
\textit{Beviset bygger på yderligere ikke-beviste sætninger, som lades op til læseren at validere}
\begin{proof}
    Antag at \((a, n) = 1\), der findes nu en \textit{linearkombination} af \(a\) og \(n\), således at \(a s + n t = 1\),
    hvilket omskrives til \(a s = 1 - n t\).
    Ud fra dette, ses det at \(a s \Mod{n} = 1\).
    Ifølge sætning \ref{modmod}, kan dette omskrives til \(\left(a \Mod{n} \cdot s \Mod{n}\right) \Mod{n} = 1\).
    Det vides at \(a < n\), da \(a \in \mathbb{Z}_n\), hvorfor \(a \Mod{n} = a\),
    derfor er \(\left(a s \Mod{n}\right) \Mod{n} = 1\),
    hvilket medfører at \(x = s \Mod{n}\) er løsning til ligningen \(a x \Mod{n} = 1\).

    Nu mangler blot entydigheden af beviset.
    Antag at begge løsningerne \(x_1, x_2 \in \mathbb{Z}_n\), er løsninger til ligningen \(a x \Mod{n} = 1\).
    Det vil altså sige \(a x_1 \Mod{n} = a x_2 \Mod{n}\),
    ud fra dette kan det konkluderes at \(x_1 \Mod{n} = x_2 \Mod{n}\)
    og da begge løsninger er mindre end \(n\), haves at \(x_1 = x_2\).
    Hvorved entydigheden er vist.

    Sætningen bevises også den anden vej:
    Antag at \(x\) er løsning til ligningen \(a x \Mod{n} = 1\),
    dette medfører at der findes et tal \(q\), så \(a x = q n + 1\).
    Ud fra dette findes en linearkombination af \(a\) og \(n\) som giver 1:
    \(x \cdot a + (-q) \cdot n = 1\).
    Ud fra en anden ikke-bevist sætning, medfører dette at \((a, n) = 1\)\cite[93]{krypto}

\end{proof}


\subsection{Bevis for RSA ved \texorpdfstring{\((m, n) \neq 1\)}{(m, n) ≠ 1}}\label{proofrsa}
\textit{Beviset bygger på yderligere ikke-beviste sætninger, som lades op til læseren at validere}
\begin{proof}
    Da \(n\) er et produkt af to primtal, vil det ene af dem være en divisor i \(m\),
    ellers kan \(n\) og \(m\) ikke have en fælles divisor større end 1.
    Det antages at \(p\) går op i \(m\) og at \(q\) ikke gør.
    Dette medfører at \((q, m) = 1\).
    Det er nu nok at vise at \(m^{ed} \Mod{p} = m \Mod{p}\)
    og \(m^{ed} \Mod{q} = m \Mod{q}\),
    for da haves at \(m^{e d} \Mod{n} = m \Mod{n} = m\) ifølge en ikke-bevist sætning.
    Da \(p | m\) fås.

    \[m^{e d} \Mod{p} = \left( m \Mod{p} \right)^{e d} \Mod{p} = 0^{e d} \Mod{p} = 0 = m \Mod{p}\]

    Sætning \ref{modmod} er anvendt ved 2. lighedstegn
    Idet \((q, m) = 1\), kan der benyttes endnu en ikke-bevist sætning.
    Som siger at \(m^{q - 1} \Mod{p} = 1\),
    da \(e d \Mod{\phi(n)} = 1\), findes der et helt tal \(k\),
    som opfylder \(e d = k \cdot \phi(n) + 1\)
    Omskrevet giver det at

    \[m^{e d} = m^{k \cdot \phi(n) + 1} = m \cdot m^{k \cdot \phi(n)} = m \cdot m^{k \cdot (p - 1) (q - 1)} = m \cdot (m^{p - 1})^{k \cdot (q - 1)}\]

    Sætning \ref{modmod} benyttes igen, hvorved det fås at:

    \begin{align*}
        m^{e d} &= \Big[ m \Mod{q} \cdot \big( m^{q - 1} \Mod{q} \big)^{k \cdot (p - 1)} \Mod{q} \Big] \Mod{q}\\
                &= \Big[ m \Mod{q} \cdot 1^{k \cdot (p - 1)} \Mod{q} \Big] \Mod{q}\\
                &= m \Mod{q}
    \end{align*}
    Hvorved det er bevist.\cite[19]{vestergaard}
\end{proof}




\newpage
\section{Python kode}
\subsection{euclid.py}
\label{file:euclid}
Filen ligger også vedhæftet i denne pdf, åbn den ved at dobbeltklikke på ikonet.
\attachfile[description={En implementering af Euclids algoritme i Python.}]{src/euclid.py}
\inputminted[python3, linenos, breaklines, frame=lines, fontsize=\footnotesize]{python}{src/euclid.py}


\subsection{prime\_factor.py}
\label{file:prime_factor}
Filen ligger også vedhæftet i denne pdf, åbn den ved at dobbeltklikke på ikonet.
\attachfile[description={Et script til at lave primtalsfaktorisering.}]{src/prime_factor.py}
\inputminted[python3, linenos, breaklines, frame=lines, fontsize=\footnotesize]{python}{src/prime_factor.py}


\subsection{phi\_func.py}
\label{file:phi_func}
Filen ligger også vedhæftet i denne pdf, åbn den ved at dobbeltklikke på ikonet.
\attachfile[description={Eulers phi-funktion, skrevet i Python.}]{src/phi_func.py}
\inputminted[python3, linenos, breaklines, frame=lines, fontsize=\footnotesize]{python}{src/phi_func.py}


\subsection{eulers\_sent.py}
\label{file:eulers_sent}
Filen ligger også vedhæftet i denne pdf, åbn den ved at dobbeltklikke på ikonet.
\attachfile[description={Et script til at udregne inverse elementer.}]{src/eulers_sent.py}
\inputminted[python3, linenos, breaklines, frame=lines, fontsize=\footnotesize]{python}{src/eulers_sent.py}
