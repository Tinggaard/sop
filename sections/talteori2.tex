%%% Kort intro til mod og primtal
%%% Noget om største fælles divisor og indbyrdes primisk
%%% Dernæst eulers funktion
%%% Så til de sætninger som er vigtige...


Talteorien beskæftiger sig med alle de hele tal \(\mathbb{Z}\).
Vi skal i løbet af dette afsnit kigge på den talteori, som ligger til grunde for at RSA virker i praksis.
% Vi skal i løbet af dette afsnit kigge på hvordan talteori ligger til grunde for, at RSA virker i praksis.

Det er ikke alle definitioner og sætninger der bliver bevist, da det ville tage for lang tid.
Alle beviser kan dog findes i \textit{Kryptografi -- fra viden til videnskab}. \cite{krypto}
\par
Først vil nogle basale begreber i talteorien blive gennemgået, hvorefter vi skal kigge på de sætninger, der ligger til grunde for RSA.

%%% rester og divisor
\subsection{Division med rest og divisor}
I stedet for at regne brøker som rationelle mængder \(\mathbb{Q}\) eller decimaltal, skal vi beskæftige os med hvordan man regner med rester.
Det vil altså sige at vi udelukkende vil beskæftige os med heltal \(\mathbb{Z}\).

Et tal \(a \neq 0\) kan enten gå op i \(b\) eller ikke gå op i \(b\).
Vi definerer at \(a\) går op i \(b\), hvis der ingen rest er, efter division. Eller sagt på en anden måde:


\begin{definition}
    \label{heldiv}
    Givet tallet \(a \neq 0\) og tallet \(b \geq a\), vil \(a\) gå op i \(b\),
    hvis der findes et heltal \(q\), som opfylder \(b = q \cdot a\).\\
    Det skrives \(a \mid b\), som betyder at \(a\) går op i \(b\).\\
    \(a\) betenes som \textit{divisoren} eller en \textit{faktor},
    % \(b\) betegnes \textit{dividenden},
    mens \(q\) kaldes for \textit{kvotienten}.\cite[70]{krypto}
\end{definition}

Er der i stedet en rest efter division, vil man kunne skrive det som \(a \nmid b\), som betyder at \(a\) ikke går op i \(b\).
% Det at arbejde med rester er også noget vi skal beskæftige os meget med.

\begin{sent}
    \label{rest}
    Givet tallet \(a > 0\) og tallet \(b\), findes der specifikke tal \(q\) og \(r\), som opfylder \(b = q \cdot a + r\), hvor \(0 \leq r < q\). % \(r\) betegnes som \textit{resten}.
\end{sent}



%%% mod
\subsubsection{Modulo}
Når man laver division med heltal, vil man ofte kigge på resten efter division.
Derfor er der naturligvis lavet en matematisk operator til at udregne rest efter division -- den kaldes for \emph{modulo}.
Skriver man f.eks. \(a \Mod{n}\), udregner man resten efter division, som angivet i sætning \ref{rest}.

\begin{definition}
    For \(m \in \mathbb{Z}\) og \(n \in \mathbb{N}\) gælder:
    \[n \mid m \iff m \Mod{n} = 0\]
    \[r = n - q m = m \Mod{n}\]%\cite[72]{krypto}
\end{definition}

Derpå, fyldes nogle vigtige sætninger om regning med modulo.
\vfill
\begin{sent}
    \label{modmod}
    For \(a, b \in \mathbb{Z}, n, t \in \mathbb{N}\)
    \begin{align}
        (a \cdot b) \Mod{n} &= \left(a \Mod{n} \cdot b \Mod{n}\right) \Mod{n}\label{eq:prodmod}\\
        a^t \Mod{n}         &= (a \Mod{n})^t \Mod{n}\label{eq:potmod}
    \end{align}
\end{sent}

\begin{proof}
    Ifølge sætning \ref{rest}, kan \(a\) og \(b\) omskrives, hvorefter udtrykket reduceres.
    \begin{align*}
(a \cdot b) \Mod{n} &= \left((q_1 n + r_1) \cdot (q_2 n + r_2)\right) \Mod{n}\\
                    &= \left( r_1 r_2 + (q_1 q_2 n + q_1 r_2 + q_2 r_1)n \right) \Mod{n}\\
                    &= r_1 r_2 \Mod{n}\\
                    &= \left(a \Mod{n} \cdot b \Mod{n}\right) \Mod{n}
    \end{align*}
    Det lange udtryk i parentesen, ved 2. lighedstegn kunne fjernes, da \(n\), var ganget ind på hvert led.
    \eqref{eq:prodmod} er nu bevist, \eqref{eq:potmod} følger, ved gentagen anvendelse af \eqref{eq:prodmod}.
\end{proof}


% \begin{eks}
%     Et par eksempler på brugen af modulo.
%     \begin{center}
%         \setlength{\tabcolsep}{20pt} % General space between cols (6pt standard)
%         \begin{tabular}{l c r}
%             \(43 \Mod{6} = 1\) & \(74 \Mod{10} = 4\) & \(58 \Mod{6} = 4\)
%         \end{tabular}
%     \end{center}
% \end{eks}


% sent 4.7 fra bogen (s 73)


%%% fælles divisor
\subsection{Fælles divisor}
En fælles divisor bruges til at sige noget om sammenhængen mellem to tal.

\begin{definition}
    For tallene \(a, b, d \in \mathbb{Z}\), betegnes \(d\) som en \emph{fælles divisor}, hvis \(d \mid a \land d \mid b\).
\end{definition}

\(a\) og \(b\) vil have et endeligt antal fælles divisorer, da et tal højere end enten \(a\) eller \(b\) naturligvis ikke er en fælles divisor (medmindre \(a\) og \(b\) begge er \(0\)).
Hvorfor der altså også vil være en største fælles divisor.
% Det kan altså konkluderes at der findes en største fælles divisor.

\begin{definition}[Største fælles divisor]
    Den største fælles divisor for tallene \(a\) og \(b\), betegnes som \((a, b)\).
\end{definition}

\begin{eks}
    Lad \(a = 8\) og \(b = 12\).\\
    \begin{tabular*}{\textwidth}{@{} l l}
        Det ses at divisorerne i \(8\) er:      & \(\pm 1, \pm 2, \pm 4, \pm 8\).\\
        Og det ses at divisorerne i \(12\) er:  &\(\pm 1, \pm 2, \pm 3, \pm 4, \pm 6, \pm 12\).\\
        Til fælles har de altså divisorerne:    & \(\pm 1, \pm 2, \pm 4\).
    \end{tabular*}
    Dermed bliver \((8, 12) = 4\).
\end{eks}

%\paragraph{Observationer}
% Det ses at fortegnet for både \(a\) og \(b\) er ubetydeligt, da kvotienterne kan være negative.

\begin{definition}[Indbyrdes primiske tal]
    Hvis \((a, b) = 1\), siges \(a\) og \(b\) at være indbyrdes primiske.
\end{definition}

Som vist, kan det være bøvlet at udregne fælles divisorer for to tal, derfor kommer nu en sætning, til at mindske vanskeligheden lidt.

\begin{sent}
    \label{commod}
    \((a, b) = (a, b \Mod{a})\)
\end{sent}

Sætning \ref{commod} bevises ikke, i stedet vil Euclids algoritme (som bygger på sætningen) blive vist.



%%% euclids alg
\subsubsection{Euclids algoritme}
Euclids algoritme er essentielt gentagen anvendelse af sætning \ref{commod}.
Algoritmen bruges til at udregne største fælles divisor for \(a, b \in \mathbb{N}\).
\begin{definition}[Euclids algoritme]
    Lad \(a, b \in \mathbb{N}\), hvor \(a \geq b\).
    Hvis \(a \mid b\), så er \((a, b) = b\).
    Hvis \(a \nmid b\), bruges følgende algoritme:
    \begin{align*}
        a   &= b q_0    + r_0   & 0 &<    r_0 < b\\
        b   &= r_0 q_1  + r_1   & 0 &\leq r_1 < r_0\\
        r_0 &= r_1 q_2  + r_2   & 0 &\leq r_2 < r_1\\
        r_1 &= r_2 q_3  + r_3   & 0 &\leq r_3 < r_2\\
        &\vdots
    \end{align*}
    Denne proces slutter, idet der findes en rest på 0.
    Det vil ske efter et endeligt antal gennemløb \(t\).
    \begin{align*}
        r_{t-2} &= r_{t-1}  q_t      + r_t   & 0 &\leq r_t < r_{t-1}\\
        r_{t-1} &= r_t      q_{t+1}  + 0
    \end{align*}
    Den sidste rest, som ikke er 0, er dermed den største fælles divisor.\cite[11]{absalg}
\end{definition}

\begin{eks}[Anvendelse af Euclids algoritme]
    Lad \(a = 1048\) og \(b = 672\).
    Ved brug af Euclids algoritme, findes da største fælles divisor.
    \begin{center}
        \setlength{\tabcolsep}{5pt} % General space between cols (6pt standard)
        \begin{tabular}{r l c r l c r}
            \((1048, 672)\) & \(= (376, 672)\) & idet & \(1048\) & \(= 1 \cdot 672\) & \(+\) & \(376\)\\
                            & \(= (376, 296)\) & --   & \(672\) & \(= 1 \cdot 376\)  & \(+\) & \(296\)\\
                            & \(= (80, 296)\)  & --   & \(376\) & \(= 1 \cdot 296\)  & \(+\) & \(80\)\\
                            & \(= (80, 56)\)   & --   & \(296\) & \(= 3 \cdot 80\)   & \(+\) & \(56\)\\
                            & \(= (24, 56)\)   & --   & \(80\)  & \(= 1 \cdot 56\)   & \(+\) & \(24\)\\
                            & \(= (24, 8)\)    & --   & \(56\)  & \(= 2 \cdot 24\)   & \(+\) & \(8\)\\
                            & \(= (0, 8)\)     & --   & \(24\)  & \(= 3 \cdot 8\)    & \(+\) & \(0\)\\
                            & \(= 8\) & & & & &
        \end{tabular}
    \end{center}
\end{eks}
Appendiks \ref{file:euclid}, viser en implementering af Euclids algoritme i Python og kan køres således
(kræver at Python er installeret\footnote{\url{https://www.python.org/downloads/}}):
\begin{minted}[fontsize=\footnotesize, bgcolor=bg]{bash}
$ python euclid.py a b
\end{minted}


%%% primtal
\subsection{Primtal}
Primtal viser sig at være interessante, i beskæftigelsen med talteori og divisorer.

\begin{definition}[Primtal]
    Primtal er heltal, større end \(1\), som kun går op i sig selv og \(1\).
\end{definition}

Der findes algoritmer, som kan vurdere med rimelig stor sikkerhed, om et gevent tal er et primtal.\cite[21]{vestergaard}
Derudover er der lavet lange lister over kendte primtal.\footnote{\url{https://primes.utm.edu/lists/small/millions/}}


% Vi skal først kigge på en sætning om primtal og division.
%
% \begin{sent}
%     For primtallet \(p\) gælder: \(p \mid a b \implies p \mid a \lor p \mid b\)
% \end{sent}



\begin{sent}% [Algebraens fundamentalsætning]
    \emph{Denne sætning er også kendt som `Algebraens fundamentalsætning''}\\
    Alle positive heltal, kan skrives som et produkt af primtal.
    Denne faktorisering er unik, hvis man ser bort fra rækkefølgen af primtallene.
    \label{algbfund}
\end{sent}

\begin{proof}
    Dette bevis bygger naturligvis på definitionen af primtal.
    Antag at \(n\) ikke er et primtal, det kan da skrives som et produkt af to tal \(n = n_1 + n_2\).
    Er \(n_1\) et primtal, forbliver det uberørt fremover, ellers faktoriseres det endnu en gang.
    Denne process fortsætter, frem til alle produkter er primtal.
\end{proof}

Appendiks \ref{file:prime_factor}, kan udregne primtalsfaktorerne for et gevent tal \(n\):
\begin{minted}[fontsize=\footnotesize, bgcolor=bg]{bash}
$ python prime_factor.py n
\end{minted}

Det ses da, at der findes uendeligt mange primtal, som følge af sætning \ref{algbfund}.
Samt at man for at tjekke om et tal \(n\) er et primtal, ikke behøver at tjekke for divisoere større end \(\sqrt{n}\).


%%% eulers phi funktion
\subsection{Eulers \texorpdfstring{\(\phi\)}{phi}-funktion}
Euler har fremskrevet en funktion, der ud fra et givet tal \(n\), kan bestemme mængden af tal, som er indbyrdes primiske med \(n\).
Først betragtes mængden \(\mathbb{Z}_n\). Som er angivet ud fra \(n \in \mathbb{N}\).
\[\mathbb{Z}_n = \{0, 1, 2, 3, \hdots, n-1 \}\]

Dette sæt repræsenterer nu alle de mulige rester efter division med \(n\), altså \(\Mod{n}\).
Der laves nu et nyt sæt bestående udelukkende af tallene, som er indbyrdes primiske med \(n\).
\[\mathbb{Z}_n^* = \{a \in \mathbb{Z}_n | a \neq 0 \land (a, n) = 1\}\]

\begin{definition}
    Eulers \(\phi\)-funktion, siges nu at være antallet af elementer i sættet \(\mathbb{Z}_n^*\).
\end{definition}

\begin{eks}
    \mbox{}\vspace*{-1.5em}
    \begin{align*}
        n &= 14 & \mathbb{Z}_{14}^*  &= \{1, 3, 5, 9, 11, 13 \} & \phi(14) &= 6\\
        n &= 9  & \mathbb{Z}_9^*     &= \{1, 2, 4, 5, 7, 8\}    & \phi(9)  &= 6
    \end{align*}
\end{eks}

Det er bevist at Eulers \(\phi\)-funktion kan defineres ud fra alle primfaktorerne i \(n\).
Det er ikke et bevis som vil blive gennemgået her.

\begin{sent}
    Givet \(p_1, p_2 \hdots p_r\), som er alle primtalsfaktorerne i \(n\).
    \[\phi(n) = n \cdot \left(1-\frac{1}{p_1}\right) \cdot \left(1-\frac{1}{p_2}\right) \cdot \hdots \cdot \left(1-\frac{1}{p_r}\right)\]
    % \[\Updownarrow\]
    % \[\phi(n) = \prod_{p \mid n}(p - 1)\]
    \[\Updownarrow\]
    \[\phi(n) = (p_1 - 1) \cdot (p_2 - 1) \cdot \hdots \cdot (p_r - 1)\]
\end{sent}

Det viser sig, at hvis \(n\) er et primtal, eller et produkt heraf, er udregningen nem.

\begin{eks}
    Lad \(p\) og \(q\) være primtal. Det haves da at.
    \begin{align}
        \phi(p)   &= p - 1\\
        %\phi(p^2) &= (p - 1)^2\\
        %\phi(p^2) &= p (p - 1)\\
        \phi(p q) &= (p - 1)(q - 1)
    \end{align}
    Dette vil vise sig nyttigt ved senere brug.
\end{eks}

Appendiks \ref{file:phi_func}, er en implementering af Eulers \(\phi\)-funktion:
\begin{minted}[fontsize=\footnotesize, bgcolor=bg]{bash}
$ python phi_func.py n
\end{minted}




%%% RSA
\subsection{Sætninger bag RSA}
Vi skal nu kigge på de sætninger, som bliver direkte anvendt i RSA metoden.
Mange af disse sætninger, bygger på sætninger, som ikke er bevist og måske endda ikke nævnt, da de alle bygger på hinanden.
Der henvises dog til appendiks \ref{proofs}, hvor udvalgte sætninger bevises.
Alternativt står beviserne naturligvis i \textit{Kryptografi -- fra viden til videnskab}. \cite{krypto}

\begin{sent}
    \label{eulerssent}
    For \(a \in \mathbb{Z}\) og \(n, t \in \mathbb{N}\) hvor \((a, n) = 1\), gælder det at:
    \[a^t \Mod{n} = a^{t \Mod{\phi(n)}} \Mod{n}\]
\end{sent}

\begin{proof}
    \(t\) angives på formen \(t = q \cdot \phi(n) + r\), hvorved \(r = t \Mod{\phi(n)}\).
    Det bevises nu at \(a^t \Mod{n} = a^r \Mod{n}\).
    \[a^t \Mod{n} = a^{q \cdot \phi(n) + r} \Mod{n}\]
    Først deles summen af potensen op i to dele.
    \[a^t \Mod{n} = (a^{\phi(n)})^q \cdot a^r \Mod{n}\]
    Ifølge sætning \ref{modmod}, kan et produkt splittes op som følger.
    \[a^t \Mod{n} = \Big[ \big( (a^{\phi(n)} \Mod{n})^q \Mod{n} \big) \cdot \big( a^r \Mod{n} \big) \Big] \Mod{n}\] %9b og 9c
    Og da \(a^{\phi(n)} \Mod{n} = 1\) (fordi \((a, n) = 1\)) \cite[90]{krypto}, fås
    \[a^t \Mod{n} = \Big[ \big( 1^q \Mod{n} \big) \cdot \big( a^r \Mod{n} \big) \Big] \Mod{n}\]
    Da \(1^q \Mod{n} = 1\), kan dette blot udelades, hvorved det ses at.
    \[a^t \Mod{n} = a^r \Mod{n}\]
\end{proof}

En anden vigtig sætning for RSA, er sætningen om inverse elementer. Den lyder som følger:

\begin{sent}
    \label{invelm}
    For \(a \in \mathbb{Z}_n\), for hvor det gælder at \((a, n) = 1\), vil \(a\) have et entydigt inverst element \(\Mod{n}\).\cite[93]{krypto}\\
    Eller sagt på en anden måde:\\
    Der vil være en entydig løsning til ligningen \(a x \Mod{n} = 1\), for \(x \in \mathbb{Z}_n\).
\end{sent}

Sætning \ref{invelm} har et langt bevis, som bygger på tidligere ikke-dokumenterede sætninger, hvorfor sætningen ikke bevises her. Det kan dog stadig findes i appendiks \ref{invelemproof}.

Python koden i appendiks \ref{file:eulers_sent}, kan udregne inverse elementer:
\begin{minted}[fontsize=\footnotesize, bgcolor=bg]{bash}
$ python eulers_sent.py a n
\end{minted}
