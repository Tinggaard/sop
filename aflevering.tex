\documentclass[a4paper, 12pt]{article}
\usepackage[danish]{babel} % Language
\usepackage[utf8]{inputenc} % Encoding
\usepackage{graphicx} % Graphics
% \usepackage{minted} % Code inclusion
% \usepackage{amsfonts} % Other letters/signs
\usepackage{lastpage} % Pagenumbering
\usepackage{fancyhdr} % Custom header/footer

% Biblografi
\usepackage{csquotes}
\usepackage[style=numeric, style=verbose-ibid]{biblatex}
\addbibresource{biblography.bib}
\DeclareNameAlias{default}{family-given}


\usepackage[pdftitle={SOP},pdfauthor={Jens Tinggaard}, hidelinks]{hyperref} %References
\usepackage[tmargin=1in,bmargin=1in,lmargin=1.25in,rmargin=1.25in]{geometry} %Margins




\pagestyle{fancy} % Set custom page layout
\fancyhf{}% to clear existing header/footer

% Set line on header and footer width
\renewcommand{\headrulewidth}{2pt}
\renewcommand{\footrulewidth}{1pt}

% Headers and footers
\lhead{Jens Tinggaard 3.E\\ Odense Tekniske Gymnasium}
\rhead{Vejledere: LEER \& SSE\\ \today}
% \lfoot{\rightmark} % Subsection
\lfoot{\leftmark} % SECTION
\rfoot{Side \thepage\ af \pageref{LastPage}}

% Navn på indholdsfortegnelse
\addto\captionsdanish{\renewcommand*\contentsname{Indholdsfortegnelse}}

% Basic info
\date{\today}
\title{SOP}
\author{Jens Tinggaard}



%%%%%%%%%%%%%%%%%%%%%%%%%%%%%%%%%%%%%%%%%%%%%%%%%%%%%%%%%%%%%%%%%%%
%%%%%%%%%%%%%%%%%%%%%%%%%%%%%%%%%%%%%%%%%%%%%%%%%%%%%%%%%%%%%%%%%%%
%%%%%%%%%%%%%%%%%%%%%%%%%%%%%%%%%%%%%%%%%%%%%%%%%%%%%%%%%%%%%%%%%%%



\begin{document}
\clearpage\maketitle
\thispagestyle{empty}
\maketitle

\begin{abstract}
her står hvad det kommer til at handle om...
\end{abstract}


\newpage
\tableofcontents


\newpage
\section{RSA}
Kort introduktion til RSA\autocite{vestergaard}

    \subsection{Modulo}
    Hvad er modulo?

    \subsection{Fælles divisor}
    Hvad er en fælles divisor?

    \subsection{Primtal}
    Hvad har primtal med det hele at gøre?

    \subsection{Eulers \texorpdfstring{$\phi$}{Lg}-funktion}
    Eulers phi funktion



\section{Hashing}
Hvad er Hashing

    \subsection{Fordelen ved hashing}
    Passwords osv.



\section{Forskkel på hashing og kryptografi}
Hvorfor man ikke blot kan nøjes med den ene

    \subsection{Alice \& Bob}
    Eksempel på brug af begge dele



\section{Hvorfor RSA er vigtig}
Hvor bliver det brugt

    \subsection{En verden uden RSA}





\newpage
\printbibliography[heading=bibintoc, title={Litteratur}]

\end{document}
