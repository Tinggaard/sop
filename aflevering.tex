\documentclass[a4paper, 12pt]{article}
\usepackage[danish]{babel} % Language
\usepackage[utf8]{inputenc} % Encoding
\usepackage{graphicx} % Graphics
% \usepackage{minted} % Code inclusion
% \usepackage{amsfonts} % Other letters/signs
\usepackage{lastpage} % Pagenumbering
\usepackage{fancyhdr} % Custom header/footer

% Biblografi
\usepackage{csquotes}
\usepackage[style=numeric, style=verbose-ibid]{biblatex}
\addbibresource{biblography.bib}
\DeclareNameAlias{default}{family-given}


\usepackage[pdftitle={SOP},pdfauthor={Jens Tinggaard}, hidelinks]{hyperref} %References
\usepackage[tmargin=1in,bmargin=1in,lmargin=1.25in,rmargin=1.25in]{geometry} %Margins




\pagestyle{fancy} % Set custom page layout
\fancyhf{}% to clear existing header/footer

% Set line on header and footer width
\renewcommand{\headrulewidth}{2pt}
\renewcommand{\footrulewidth}{1pt}

% Headers and footers
\lhead{Jens Tinggaard 3.E\\ Odense Tekniske Gymnasium}
\rhead{Vejledere: LEER \& SSE\\ \today}
% \lfoot{\rightmark} % Subsection
\lfoot{\leftmark} % SECTION
\rfoot{Side \thepage\ af \pageref{LastPage}}

% Navn på indholdsfortegnelse
\addto\captionsdanish{\renewcommand*\contentsname{Indholdsfortegnelse}}

% Basic info
\date{\today}
\title{SOP}
\author{Jens Tinggaard}


\setlength{\parindent}{0em} % Identeringsstørrelse
\setlength{\parskip}{1em} % Paragraf afstand



%%%%%%%%%%%%%%%%%%%%%%%%%%%%%%%%%%%%%%%%%%%%%%%%%%%%%%%%%%%%%%%%%%%
%%%%%%%%%%%%%%%%%%%%%%%%%%%%%%%%%%%%%%%%%%%%%%%%%%%%%%%%%%%%%%%%%%%
%%%%%%%%%%%%%%%%%%%%%%%%%%%%%%%%%%%%%%%%%%%%%%%%%%%%%%%%%%%%%%%%%%%



\begin{document}
\clearpage\maketitle
\thispagestyle{empty}
\maketitle

\begin{abstract}
her står hvad det kommer til at handle om...
\end{abstract}

% Indsæt billede til forside?


\newpage
\tableofcontents


\newpage
\section{Hvad er RSA}
Uden nødvendigvis at være klar over det, er vi alle afhængige af RSA, det er en helt fundamental ting af vores hverdag.
RSA er grunden til at man kan handle på nettet, uden at få stjålet sine kreditkortoplysninger.
RSA-krytografi ligger til grunde for, at man ikke blot kan lave online identitetstyveri.
\autocite{vestergaard}
\par
RSA er en metode, brugt til at kryptere data, døbt efter sine tre stifere:
\textit{Ron Rivers}, \textit{Adi Shamir} og \textit{Len Adleman}, tilbage i 1977.

    \subsection{Modulo}
    Hvad er modulo?

    \subsection{Fælles divisor}
    Hvad er en fælles divisor?

    \subsection{Primtal}
    Hvad har primtal med det hele at gøre?

    \subsection{Eulers \texorpdfstring{$\phi$}{Lg}-funktion}
    Eulers phi funktion



\section{Hashing}
Hvad er Hashing

    \subsection{Fordelen ved hashing}
    Passwords osv.



\section{Forskkel på hashing og kryptografi}
Hvorfor man ikke blot kan nøjes med den ene

    \subsection{Alice \& Bob}
    Eksempel på brug af begge dele



\section{Hvorfor RSA er vigtig}
Hvor bliver det brugt

    \subsection{En verden uden RSA}





\newpage
\printbibliography[heading=bibintoc, title={Litteratur}]

\end{document}
