\documentclass[a4paper, 12pt]{article}
\usepackage[danish]{babel} % Language
\usepackage[utf8]{inputenc} % Encoding
\usepackage{graphicx} % Graphics
% \usepackage{minted} % Code inclusion
% \usepackage{amsfonts} % Other letters/signs
\usepackage{lastpage} % Pagenumbering
\usepackage{fancyhdr} % Custom header/footer
\usepackage{amsthm} % Proofs
\usepackage{amssymb} % Black square
\usepackage{mdframed} % Boxing lemmas and such
\usepackage{float} % Custom examples
\usepackage[pdftitle={Sikkerhed bag login-formular på en hjemmeside},pdfauthor={Jens Tinggaard}, hidelinks, bookmarks=true]{hyperref} %References
\usepackage[tmargin=1in,bmargin=1in,lmargin=1.25in,rmargin=1.25in]{geometry} %Margins

% Biblografi
\usepackage{csquotes}
\usepackage[style=numeric, style=verbose-ibid]{biblatex}
\addbibresource{biblography.bib}
\DeclareNameAlias{default}{family-given}


\pagestyle{fancy} % Set custom page layout
\fancyhf{}% to clear existing header/footer
% Set line on header and footer width
\renewcommand{\headrulewidth}{2pt}
\renewcommand{\footrulewidth}{1pt}



\renewcommand\qedsymbol{$\blacksquare$} % Black QED

\newtheoremstyle{break}% name
  {3pt}%         Space above, empty = `usual value'
  {3pt}%         Space below
  {\itshape}% Body font
  {}%         Indent amount (empty = no indent, \parindent = para indent)
  {\bfseries}% Thm head font
  {}%        Punctuation after thm head
  {\newline}% Space after thm head: \newline = linebreak
  {}%         Thm head spec


\theoremstyle{break}
\newtheorem{lemma}{Lemma} % Lemma kommando
\newenvironment{boxframed}
  {\begin{mdframed}\begin{lemma}}
  {\end{lemma}\end{mdframed}}

% \newtheorem{theorem}{Theorem}
% \newtheorem{claim}[theorem]{Claim}
% \newtheorem{proposition}[theorem]{Proposition}
% \newtheorem{lemma}[theorem]{Lemma}
% \newtheorem{corollary}[theorem]{Corollary}
% \newtheorem{conjecture}[theorem]{Conjecture}
% \newtheorem*{observation}{Observation}
% \newtheorem*{example}{Example}
% \newtheorem*{remark}{Remark}

% Eksempel float
\floatstyle{plain}
\newfloat{eks}{h}{eks}
\floatname{eks}{Eksempel}



% Headers and footers
\lhead{Jens Tinggaard 3.E\\ Odense Tekniske Gymnasium}
\rhead{Vejledere: LEER \& SSE\\ \today}
% \lfoot{\rightmark} % Subsection
\lfoot{\leftmark} % SECTION
\rfoot{Side \thepage\ af \pageref{LastPage}}

% Navn på indholdsfortegnelse
\addto\captionsdanish{\renewcommand*\contentsname{Indholdsfortegnelse}}

% Basic info
\date{\today}
\title{Sikkerhed bag login-formular på en hjemmeside}
\author{Jens Tinggaard}


% \setlength{\parindent}{0em} % Identeringsstørrelse
% \setlength{\parskip}{1em} % Paragraf afstand
% Bruges med \par


%%%%%%%%%%%%%%%%%%%%%%%%%%%%%%%%%%%%%%%%%%%%%%%%%%%%%%%%%%%%%%%%%%%
%%%%%%%%%%%%%%%%%%%%%%%%%%%%%%%%%%%%%%%%%%%%%%%%%%%%%%%%%%%%%%%%%%%
%%%%%%%%%%%%%%%%%%%%%%%%%%%%%%%%%%%%%%%%%%%%%%%%%%%%%%%%%%%%%%%%%%%



\begin{document}
\clearpage\maketitle
\thispagestyle{empty}
\maketitle

\begin{abstract}
her står hvad det kommer til at handle om...
\end{abstract}

% Indsæt billede til forside?


\newpage
\tableofcontents


\newpage
\section{Hvad er RSA}
Uden nødvendigvis at være klar over det, er vi alle afhængige af RSA, det er en helt fundamental ting af vores hverdag.
RSA er grunden til at man kan handle på nettet, uden at få stjålet sine kreditkortoplysninger.
RSA-krytografi ligger også til grunde for, at man ikke blot kan lave online identitetstyveri.\autocite{vestergaard}
\\
\\
RSA er en metode, brugt til at kryptere data, døbt efter sine tre stifere:
\textit{Ron Rivers}, \textit{Adi Shamir} og \textit{Len Adleman}, tilbage i 1977.\autocite{vestergaard}
Metoden er baseret på antagelsen om, at det er svært at primtalsfaktorisere et stort tal.
Talene anvendt ligger typisk i intervallet \(2^{1024}\) til \(2^{4096}\).
Hvilket svarer til tal mellem \(\approx1.798^{308}\) og \(\approx1.044^{1233}\) i decimaltal.\autocite{MANGLER}

    \subsection{Modulo}
    Hvad er modulo?

    \subsection{Fælles divisor}
    Hvad er en fælles divisor?


    \begin{boxframed}
        Linjens ligning \(y=ax+b\)
    \end{boxframed}

    \begin{proof}
        Ovenstående er sandt, fordi jeg siger det!
    \end{proof}

    \subsection{Primtal}
    Hvad har primtal med det hele at gøre?

    \subsection{Eulers \texorpdfstring{\(\phi\)}{Lg}-funktion}
    Eulers phi funktion



\section{Hashing}
Hashing er en metode, brugt til at omdanne en tekststreng til en anden tekststreng, med en fikseret længde.
Den nye tekststreng vil være pseudorandom - den vil fremstå tilfædigt, men faktisk være regelmæssig, baseret på inputtet.
Til forskel fra RSA skal man ikke skal generere nogle tal førend metoden bruges, hvilket medfører at der altid bliver genereret det samme output, ved det samme input.
\\
Hashing en også envejsfunktion, modsat RSA, hvor man jo både kan kryptere og dekryptere data.
Med hashing kan man altså ikke gå baglæns, hvis man kun har et hash af en tekst.


    \subsection{Fordelen ved hashing}
    Det at hashing kun går en vej, virker måske lidt nytteløst, da man ikke længere er i stand til at finde ud af hvad man har hashet.
    I stedet bruger man hashing til at gemme kodeord, sådan at hvis nogle kodeord bliver lækket, er de stadig beskyttede.
    Dette kan lade sig gøre, da hashing altid giver det samme hash ved brug af samme input.
    Det vil altså sige at hvis man hasher \(abc\), vil man få det samme output hver gang.
    Da et hash er pseudorandom, vil det ikke sige noget om hvad det tidligere har været - der er som sådan ikke noget mønster i metoden.
    Det ses på eksempel \autoref{tab:hash}.


    \begin{eks}
        \centering
        \texttt{MD5(abc) = 900150983cd24fb0d6963f7d28e17f72}\\
        \texttt{Md5(Abc) = 35593b7ce5020eae3ca68fd5b6f3e031}

        \caption{Hashing eksempel med MD5}
        \label{tab:hash}
    \end{eks}

    \noindent
    Måden man anvender dette på, er at hashe et kodeord inden det bliver puttet ind i databasen.
    Når så brugeren forsøger at logge ind, hasher man igen det indtastede kodeord og hvis det stemmer over ens, med det liggende i databasen, kan man formode at brugeren har indtastet det rigtige kodeord.




\section{Forskkel på hashing og kryptografi}
Hvorfor man ikke blot kan nøjes med den ene

    \subsection{Alice \& Bob}
    Eksempel på brug af begge dele



\section{Hvorfor RSA er vigtig}
Hvor bliver det brugt

    \subsection{En verden uden RSA}





\newpage
\printbibliography[heading=bibintoc, title={Litteratur}]

\end{document}
